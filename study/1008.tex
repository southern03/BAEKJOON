\documentclass[a4paper,12pt]{article}
\usepackage{listings}
\usepackage{xcolor}
\usepackage{array}

\lstset{
  basicstyle=\ttfamily\footnotesize,
  tabsize=4,
  showstringspaces=false,
  numbers=left,
  numberstyle=\tiny\color{gray},
  keywordstyle=\color{blue},
  commentstyle=\color{green!50!black},
  stringstyle=\color{red},
  breaklines=true
}

\title{C vs C++ 출력 형식 정리}
\author{}
\date{}

\begin{document}
\maketitle

\section*{✅ C 스타일 (\texttt{printf})}

\begin{lstlisting}[language=C]
// 소수점 9자리까지 출력
printf("%.9lf\n", (double)A / B);

// 정수 10칸 오른쪽 정렬
printf("%10d\n", 42);

// 5자리, 빈 칸을 0으로 채움
printf("%05d\n", 42);
\end{lstlisting}

\section*{✅ C++ 스타일 (\texttt{cout} + <iomanip>)}

\begin{lstlisting}[language=C++]
#include <iostream>
#include <iomanip>
using namespace std;

int main() {
    int A = 42;
    double B = 3.1415926535;

    // 소수점 9자리까지 출력
    cout << fixed << setprecision(9) << B << '\n';

    // 정수 10칸 오른쪽 정렬
    cout << setw(10) << A << '\n';

    // 5자리, 빈칸을 0으로 채움
    cout << setw(5) << setfill('0') << A << '\n';

    return 0;
}
\end{lstlisting}

\section*{🔎 기능별 비교표}

\renewcommand{\arraystretch}{1.4}
\begin{tabular}{|>{\raggedright}m{3.2cm}|>{\raggedright}m{5cm}|>{\raggedright\arraybackslash}m{6cm}|}
\hline
\textbf{기능} & \textbf{C (\texttt{printf})} & \textbf{C++ (\texttt{cout})} \\
\hline
소수점 자릿수 & \texttt{\%.9lf} & \texttt{fixed << setprecision(9)} \\
\hline
과학적 표기 & \texttt{\%e} & \texttt{scientific} \\
\hline
기본 표기 & \texttt{\%f} & \texttt{defaultfloat} \\
\hline
출력 폭 지정 & \texttt{\%10d} & \texttt{setw(10)} \\
\hline
0 패딩 & \texttt{\%05d} & \texttt{setw(5) << setfill('0')} \\
\hline
문자열 출력 & \texttt{\%s} & \texttt{cout << string} \\
\hline
\end{tabular}

\section*{📍 정리}
\begin{itemize}
  \item \textbf{C}: 문자열 포맷(\%d, \%lf) 기반, 저수준/빠름
  \item \textbf{C++}: 스트림 조작자(\texttt{setw}, \texttt{setprecision}) 기반, 가독성/유연성 좋음
  \item \textbf{백준/알고리즘 문제 풀이}: \texttt{cout << fixed << setprecision(n)}을 거의 표준처럼 사용
\end{itemize}

\end{document}
